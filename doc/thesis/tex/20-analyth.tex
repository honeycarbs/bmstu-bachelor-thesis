\chapter{Аналитический раздел}
\section{Категоризация эмоциональных данных}
Одной из главных проблем в исследованиях, связанных с определением эмоционального состояния диктора по голосу, является отсутствие четкого определения эмоции. Подход к классификации эмоций влияет на процесс аннотирования. Сегодня широко используются три подхода к категоризации эмоциональных данных: дискретный, многомерный и гибридный.
\subsection{Дискретное пространство эмоций}
Дискретный подход основан на выделении фундаментальных (базовых) эмоций, сочетания которых порождают разнообразие эмоциональных явлений. Разные авторы называют разное число таких эмоций -- от двух до десяти. П.~Экман на основе изучения лицевой экспрессии выделяет пять базовых эмоций: гнев, страх, отвращение, печаль и радость. Первоначальная версия 1999 года также включала <<удивление>> \cite{Ekman1972, Ekman1992}. Р.~Плутчик \cite{Plutchik1980} выделяет восемь базисных эмоций, деля их на четыре пары, каждая из которых связана с определенным действием: страх, уныние, удивление и т.~д. 

%На рисунке \ref{fig:emo-atlas} представлена схема классификации эмоций, предложенная П.~Экманом. 

На сегодняшний день существование базовых эмоций ставится под сомнение. Теория встречает ряд концептуальных проблем, таких как, например, эмпирическое определение набора базовых эмоций или критерии синхронизации эмоциональных реакций. Однако, многие решения в области автоматического детектирования эмоций основаны на дискретной модели эмоциональной сферы. Например, решение компании <<Affectiva>>. \cite{Affectica}

\subsection{Многомерное пространство эмоций}
Многомерное пространство представляет собой эмоции в координатном многомерном пространстве. В качестве ее источника рассматривают идею В. Вундта о том, что многогранность чувств
человека можно описать с помощью трех измерений: удовольствие-неудовольствие, расслабление-напряжение, возбуждение-успокоение. Вундт заключил, \cite{Вундт1984} что эти измерения охватывают все разнообразие эмоциональных состояний. Данные для этой теории были получены с помощью метода интроспекции.

Эмоциональная сфера представляется как многомерное пространство, образованное некоторым
количеством осей координат. Оси задаются полюсами первичных характеристик эмоций. Отдельные эмоции -- это точки, местоположение которых в <<эмоциональном>> пространстве определяется степенью выраженности этих параметров.

Один из примеров описываемого подхода -- модель Дж. Рассела. В ней водится двумерный базис, в котором каждая эмоция характеризуется валентностью (\textit{англ. valence}) и интенсивностью (\textit{англ. arousal}). Измерение валентности отражает то,
насколько хорошо человек ощущает себя на уровне субъективного переживания от максимального неудовольствия до максимального удовольствия. Измерение активации связано с
субъективным чувством энергии и ранжируется в диапазоне от дремоты до бурного возбуждения. Такой подход используется, например, в наборе данных <<RECOLA>> \cite{RECOLA}.

Аналогично вопросу о количестве эмоций в дискретной модели, вопрос о количестве измерений остается открытым. Использование только двух критикуется на том основании, что они не позволяют устанавливать различия между отдельными эмоциональными состояниями (например, страх, гнев, ревность, презрение и др. имеют отрицательную валентность и высокую активацию).

\subsection{Гибридное пространство эмоций}
Гибридная модель представляет собой комбинацию дискретной и многомерной модели. Примером такой модели являются <<Песочные часы эмоций>>, предложенные Камбрией, Ливингстоном и Хуссейном. \cite{hourglass} 

Согласно этой классификации, в отдельной области $n$-мерного эмоционального пространства различия между эмоциями могут определяться в терминах измерений, имеющих отношение к этой области. Эмоции могут быть сопоставимы по измерениям внутри и вне категорий, и каждая категория может иметь свои отличительные признаки. \cite{Russell2003} Каждое измерение характеризуется шестью уровнями силы, с которой выражены эмоции. Данные уровни обозначаются набором из двадцати четырех эмоций. Поэтому совершенно любая эмоция может рассматриваться как и фиксированное состояние, так и часть пространства, связанная с другими эмоциями нелинейными отношениями. 

\section{Информативные признаки эмоциональных состояний}
\subsection{Классификация признаков}
На эффективность классификации значительное влияние оказывает выбор набора информативных признаков, характеризующих речь. В целом, признаки можно разбить на две категории: спектральные (акустические) и лингвистические (просодические). \cite{schuller2011recognising}

Лингвистические признаки (мелодика речи, интенсивность, ритм, тембр, сила голоса) характеризуют содержательный аспект речи. При вычислении акустических признаков речевой поток рассматривается как некоторый квазистационарный процесс.

При вычислении спектральных признаков речевой сигнал представляется в виде дискретной последовательности
цифровых значений амплитуды речевой волны, подвергается спектральному анализу. Эти характеристики
могут быть условно разделены на 9 групп. \cite{розалиев2007построение}
\begin{enumerate}
	\item Средние значения спектра анализируемого речевого сигнала.
	\item Нормализованные средние значения спектра.
	\item Относительное время пребывания сигнала в полосах спектра.
	\item Нормализованное время пребывания сигнала в полосах спектра.
	\item Медианные значения спектра речи в полосах.
	\item Относительная мощность спектра речи в полосах.
	\item Величины вариации огибающей спектра речи.
	\item Нормализованные величины вариации огибающих спектра речи.
	\item Значения коэффициентов кросскорреляции спектральных огибающих между полосами спектра.
\end{enumerate}
Признаки 1-7 отражают особенности речевых трактов у разных лиц. Интенсивность сигнала определяют признаки 1, 2. Признаки 7, 8 связаны  с динамикой перестройки артикуляционных органов речи говорящего. Признак 9 характеризует синхронность органов речи говорящего.

Выбор категории зависит от решаемой задачи. Например, в системах где речь соответствует заранее определенному сценарию (словарь голосовых команд) в наборе присутствуют в основном спектральные характеристики, в то время как при анализе слитной речи выбираются лингвистические характеристики. Для анализа эмоционального компонента речи, не касающегося её смысла, используются как просодические признаки (громкость, высота, ритм), так и спектральные.
\subsection{Характеристика речи эмоционального состояния}
	
\section{Извлечение информативных признаков}
Для создания и обучения модели, которая в будущем сможет предсказывать эмоции по речевому сигналу, необходимо извлечь информативные признаки из речи и перевести их в количественные показатели. 

Пусть на вход модели поступает аудиосигнал. Для того, чтобы привести его в вид, который будет использован алгоритмом распознавания (оцифровать), сигнал делится на фреймы и свойства определяются в каждом фрейме. Размер фрейма выбирается от 20 до 40 мс, поскольку считается, что речевой сигнал на этом промежутке стационарный. Каждая точка, как правило, перекрывается дважды. \cite{mfcc-steps} Таким образом, речевой сигнал представляется в виде \ref{eq:frames}:
\begin{equation}\label{eq:frames}
	x_i(n),\;0 \leq n < N,
\end{equation}
где $N$ – размер фрейма, $x_i(n)$ -- $i$-ый фрейм. \\


%Методы сравнения звуковых сигналов могут быть разделены на две категории в зависимости от области обработки: частотные и временные. Методы обработки во временной области заключаются в определении характерных точек речевого сигнала. \cite{розалиев2008предпосылки} 	Сравнение звуковых сигналов во временной области будет непроизводительным и малоэффективным, поскольку из-за шумов и смещений нулевого уровня возникает возникает неоднозначность выделения характерных точек. Соответственно, следует использовать частотные методы сравнения сигналов.
%
%При использовании частотных методов речь следует представлять с помощью признаков, полученных в частотной области. Речевые сигналы имеют специфический частотный состав и занимают характерные спектральные области. Использование методов в частотной области позволяет обрабатывать речевые сигналы с достаточно высокой точностью.
%
%Для получения признаков, используемых в распознавании речи используется дискретное преобразование Фурье (ДПФ) 


% К этому процессу предъявляются следующие требования:
%\begin{itemize}
%	\item количество признаков используемых для обучения модели следует минимизировать
%	\item content...
%\end{itemize}


%Для анализа эмоционального компонента речи, не касающегося её смысла, используются просодические признаки: громкость, высота, ритм, придаваемые речи. Оцениваются и спектральные характеристики аудиоматериала. 

%В целом, для выделения признаков могут использоваться следующий математический аппарат: метод главных компонент (PCA), мел-кепстральные коэффициенты (MFCC), кодирование с линейным прогнозированием (LPC), перцептивное линейное предсказание (PLP).
\subsection{Метод главных компонент}
\subsection{Мел-кепстральные коэффициенты}
%Воспроизводимые человеком звуки определяются формой голосового тракта, включая язык, зубы и т. д. Точное представление о воспроизводимой фонеме можно получить, определив огибающую спектра, которой описывается форма голосового тракта.
Для представления огибающей спектра, которой описывается форма голосового тракта были введены мел-кепстральные коэффициенты (\textit{англ. MFCC}). \cite{mfcc}

Шкала Мел (рисунок \ref{fig:mel-hz}) соотносит воспринимаемую частоту или высоту чистого тона (мел) с фактической измеренной частотой (Гц). Люди гораздо лучше различают небольшие изменения высоты звука на низких частотах, чем на высоких. \cite{mel} 
\begin{figure}[H]
	\centering
	\begin{tikzpicture}
		\begin{axis}[
			legend pos = north west,
			xmin=0, xmax=10000,
			ymin=0, ymax=3200,
			/pgf/number format/.cd,
			use comma,
			1000 sep={},
			xlabel= Фактическая измеренная частота (Гц),
			ylabel= Высота чистого тона (мел) ,
			grid = both,
			grid style = {dashed, lightgray!35},
			xtick distance = 500,
			ytick distance = 200,
			width = 0.98\textwidth,
			tick label style={font=\scriptsize},
			scaled ticks=false,
			height=0.35\textheight,]
			\addplot[
			red,
			semithick,
			domain=-0:10000,
			] {1127.01048 * ln(1 + x /700)};
		\end{axis}
	\end{tikzpicture}
	\caption{График зависимости частоты от мел}
	\label{fig:mel-hz}
\end{figure}
Формула перевода фактических частот в высоту чистого тона описывается согласно \ref{eq:mel-hz}:

\begin{equation}\label{eq:mel-hz}
	m = 1127,01048 \cdot \ln(1 + \cfrac{f}{700}),
\end{equation}
где где $f$ -- фактическая измеренная частота (Гц), $m$ -- чистого тона (мел).
\noindent Обратное преобразование из мел в фактическую частоту вычисляется согласно формуле \ref{eq:hz-mel}:
\begin{equation}\label{eq:hz-mel}
	f = 700(e^{m/1127,01048} - 1). 
\end{equation}
%Для вычисления мел-частотных кепстральных коэффициентов необходимо разделить исходный сигнал на фреймы. Размер фрейма выбирается от 20 до 40 мс, поскольку считается, что речевой сигнал на этом промежутке стационарный. Перекрытие, как правило, берут равным 50\%. \cite{mfcc-steps} Таким образом, речевой сигнал представляется в виде \ref{eq:mfcc-sig}:
%\begin{equation}\label{eq:mfcc-sig}
%	x_i(n),\;0 \leq n < N,
%\end{equation}
%где $N$ – размер фрейма или длина окна, $x_i(n)$ -- $i$-ый фрейм. \\
%Речевой сигнал конечен и не является периодическим, поэтому при применении преобразования Фурье на его концах появляется эффект утечки. Для снижения влияния этого эффекта на результат каждый кадр умножается на оконную функцию Хемминга согласно \ref{eq:mfcc-hamming}:
%\begin{equation}\label{eq:mfcc-hamming}
%	w(n) = 0,54 - 0,46 \cdot \cos\left(\cfrac{2\pi n}{N - 1}\right),\; 0 \leq n \leq N - 1.
%\end{equation}
%Следующим шагом применяется дискретное преобразование Фурье согласно \ref{eq:mfcc-dft}:
%\begin{equation}\label{eq:mfcc-dft}
%	X_i(k) = \sum_{n=0}^{N-1}x_i(n)w(n)e^{\cfrac{2\pi n}{N}kn}, 0 \leq k \leq N,
%\end{equation}
%где $i$ -- номер фрейма.
\subsection{Кодирование с линейным прогнозированием}
\subsection{Перцептивное линейное предсказание}

%\subsection{Характеристики речи при различных эмоциональных состояниях}

%\subsection{Извлечение информативных признаков}

%<...>
%
%\textbf{Гнев} \\
%
%\textit{Отвращение} \\
%
%\textit{Страх} \\
%
%\textit{Радость} \\
%
%\textit{Грусть} \\

%\subsection{Выделение признаков из речевого сигнала}
%\subsection{Лингвистические признаки}
%В зависимости от решаемой задачи их относительная эффективность может быть различной. Системы, в которых речь соответствует заранее определенному сценарию (словарь
%голосовых команд), на первый план выходят акустические параметры, в то время как при ра-
%боте со спонтанной речью (слитная речь) роль лингвистических признаков может оказаться
%весьма существенной [3]

% просодические (лингвистические) и акустические (спектральные). 
%Важнейшим звеном системы автоматического детектирования эмоций
%по голосу диктора является выделение оптимального набора информативных
%признаков, коррелированных с эмоциональными состояниями. Выбор информа-
%тивных признаков оказывает значительное влияние на эффективность класси-
%фикации. 
\section{Существующие наборы речевых данных}
\section{Классификаторы, используемые в SER}
%\section{Представление адудиосигнала}
\section{Постановка задачи}