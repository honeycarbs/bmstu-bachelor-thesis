\part*{ЗАКЛЮЧЕНИЕ}
\addcontentsline{toc}{part}{\textbf{ЗАКЛЮЧЕНИЕ}}
В рамках настоящей работы был разработан и реализован метод распознавания эмоций по звучащей речи. Все поставленные цели были выполнены.

Были проанализированы русскоязычные и иностранные корпуса эмоциональной речи. Русскоязычных корпусов с подходящей эмоциональной разметкой всего 2: RUSLANA и DUSHA. Корпуса RUSLANA нет в открытом доступе, поэтому для обучения классификатора был выбран корпус DUSHA.

Были проанализированы как просодические, так и спектральные признаки, характеризующие эмоцию в речи. Для обучения классификатора были использованы мел-кепстральные коэффициенты, поскольку они более устойчивы к шуму и содержат достаточно широкий набор информации о речи.

Был проведен обзор классификаторов, чаще всего применяющихся в системах распознавания эмоций в речи: скрытой марковской модели и искусственной нейронной сети.

При реализации метода было использовано дискретное пространство эмоций, включающее в себя 4 эмоции: <<злость>>, <<радость>>, <<грусть>> и <<нейтраль>>. 

Классификатор, реализованный в рамках метода, на наборе данных, объем которых составляет 6000 аудиозаписей, показал наиболее высокое качество классификации у эмоций <<ярость/раздражение>> и <<позитив>>, наиболее низкое -- у классов <<нейтраль>> и <<грусть>>,  причем наблюдаемая разница составила почти 1.5 раза. При изменении размера обучающей выборки было выяснено, что положительная тенденция наблюдается у каждого класса разметки. Можно предположить, что для качественной классификации эмоций, имеющих более выраженные признаки, требуется меньший размер обучающей выборки.


В дальнейших исследованиях планируется обучение классификатора на собственном корпусе данных, содержащем информацию не только об эмоции, которую выражал говорящий, но и об интонационном контуре, которым обладает высказывание. За счет связи интонационного контура и некоторых эмоций (например, ИК-6 и эмоции удивления) можно повысить качество классификации. 