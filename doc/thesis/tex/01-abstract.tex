\part*{РЕФЕРАТ}
%\thispagestyle{empty}
\addcontentsline{toc}{part}{\textbf{РЕФЕРАТ}}

Расчетно--пояснительная записка \pageref{LastPage} с., \totalfigures\ рис., \totaltables\ табл., \thetotalbibentries\ ист, 1 прил.

Объектом разработки является метод идентификации эмоций по звучащей речи. Цель работы -- спроектировать и реализовать метод распознавания эмоций по звучащей речи на основе скрытой марковской модели. 

Система идентификации эмоций обучена на корпусе данных DUSHA, разработанного в 2022 году и на данный момент не использованного для систем распознавания эмоций в речи. F-мера разработанного классификатора в среднем  $\approx$35\%, по каждому классу отдельно -- от 28\% до 43\%. 

Знания об эмоции, которую испытывает человек в данный момент, могут быть использованы во многих сферах HCI, начиная от улучшения качества обслуживания и повышения покупательской способности и заканчивая повышением эффективности коммуникации и оказанием психологической помощи. 