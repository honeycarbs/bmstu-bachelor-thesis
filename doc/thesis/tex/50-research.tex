\chapter{Исследовательский раздел}

\section{Предварительная обработка обучающего набора данных}
Перед составлением тренировочной и обучающих выборок корпус DUSHA был проанализирован на предмет процентного соотношения каждой классов в разметке. Распределение классов разметки по доменам представлено в таблице \ref{tab:stats}.

\begin{table}[H]
	\centering
	\caption{Распределение классов разметки в корпусе DUSHA}\label{tab:stats}
	\begin{tabular}{|c|R|E|R|E|}
		\hline
		\textit{Домен} & \multicolumn{2}{c|}{\textit{Crowd}} & \multicolumn{2}{c|}{\textit{Podcast}} \\ \hline
		\textit{Эмоция} & \textit{Количество, шт.} & \textit{Доля, \%} & \textit{Количество, шт.} & \textit{Доля, \%} \\ \hline
		positive & 15446 & 9.40 & 5909 & 6.53 \\ \hline
		sad & 23316 & 14.18 & 1170 & 1.29 \\ \hline
		angry & 17120 & 10.41 & 2057 & 2.27 \\ \hline
		neutral & 106850 & 65.00 & 81104 & 89.66 \\ \hline
		other & 1655 & 1.01\% & 222 & 0.25 \\ \hline
	\end{tabular}
\end{table}
Из таблицы \ref{tab:stats} видно, что  классы в разметке неравномерно распределены. Поэтому для обучения и проверки скрытой марковской модели был использован не весь набор данных DUSHA, а его часть. Из файлов разметки были извлечены все аудиофайлы, длина которых не превышала 3 секунды и разметка которых не содержала значения <<other>>. В обучающую выборку было включено 1500 аудиофайлов каждого класса разметки. Объем данных обучающей выборки, которая подается на вход классификатору, представлен в таблице \ref{tab:stats-my}.

\begin{table}[H]
	\centering
	\caption{Объем данных обучающей выборки}\label{tab:stats-my}
	\renewcommand{\arraystretch}{1.3}
	\begin{tabular}{|c|F|F|F|}
		\hline
		\textit{Подгруппа} & \textit{Всего} & \textit{Тренировочная выборка} & \textit{Тестовая выборка} \\ \hline
		angry & 1 ч. 02 мин. 47 сек.  & 50 мин. 05 сек. & 12 мин. 41 сек. \\ \hline
		neutral & 1 ч. 02 мин. 23 сек. & 50 мин. 02 сек. & 12 мин. 20 сек. \\ \hline
		positive & 1 ч. 02 мин. 01 сек.& 50 мин. 32 сек. & 12 мин. 29 сек. \\ \hline
		sad & 1 ч. 02 мин. 53 сек. & 51 мин. 57 сек. & 12 мин. 55 сек. \\ \hline
	\end{tabular}
\end{table}

\section{Результат классификации и его оценка}
Результат распознавания на тренировочной выборке представлен в виде таблицы \ref{tab:confusion-matrix}. 
\begin{table}[H]
	\centering
	\caption{Матрица несоответствий}\label{tab:confusion-matrix}
	\renewcommand{\arraystretch}{1.3}
	\begin{tabular}{|E|E|E|E|E|}
		\hline
		\textit{Экспертная} & \multicolumn{4}{c|}{\textit{Оценка классификатора}} \\ \cline{2-5}
		\textit{оценка} & \textit{angry} & \textit{neutral} & \textit{positive} & \textit{sad} \\ \hline
		\textit{angry} & 0.443 & 0.062 & 0.359 & 0.136 \\ \hline
		\textit{neutral} & 0.177 & 0.118 & 0.448 & 0.258 \\ \hline
		\textit{positive} & 0.308 & 0.058 & 0.503 & 0.131 \\ \hline
		\textit{sad} & 0.28 & 0.073 & 0.346 & 0.301 \\ \hline
	\end{tabular}
\end{table}

По матрице несоответствий можно рассчитать следующие важные для оценки параметры: TP, FP, TN, FN. Значения и расшифровка этих параметров представлены в таблице \ref{tab:tpfn}.
\begin{table}[H]
	\centering
	\caption{Значения TP, TN, FP, FN каждого класса разметки}\label{tab:tpfn}
	\renewcommand{\arraystretch}{1.3}
	\begin{tabular}{|O|P|P|P|P|}
		\hline
		\multirow{3}{*}{\textit{Класс}} & \textit{TP} & \textit{TN} & \textit{FP} & \textit{FN} \\
		\cline{2-5}
		& \textit{Истинно} & \textit{Истинно} & \textit{Ложно} & \textit{Ложно} \\
		& \textit{положительные} & \textit{отрицательные} & \textit{положительные} & \textit{отрицательные} \\
		\hline
		\textit{angry} & 533 & 4095 & 920 & 467 \\
		\hline
		\textit{neutral} & 141 & 4373 & 780 & 1059 \\
		\hline
		\textit{positive} & 603 & 4690 & 1206 & 1007 \\
		\hline
		\textit{sad} & 361 & 4491 & 462 & 799 \\
		\hline
		\end{tabular}
\end{table}

Для оценки производительности классификатора на основе матрицы несоответствий также можно учесть несколько показателей оценки, таких как точность (\textit{англ. precision}), полнота (\textit{англ. recall}) и F1-мера.

Точность системы в пределах класса -- это доля элементов действительно принадлежащих данному классу относительно всех элементов которые система отнесла к этому классу. Рассчитывается согласно \ref{eq:precision}:
\begin{equation}\label{eq:precision}
	\mathrm{Precision_c} = \cfrac{A_{c,\,c}}{\displaystyle\sum_{i = 1}^{n}A_{c,\,i}},
\end{equation}
где $A$ -- матрица несоответствий, $c$ -- индекс класса для которого вычисляется точность в матрице несоответствий. Значение точности для каждого класса разметки представлено в таблице \ref{tab:pres}.

\begin{table}[H]
	\centering
	\caption{Значение точности для каждого класса разметки}\label{tab:pres}
	\begin{tabular}{|E|Q|Q|Q|Q|Q|}
		\hline
		\textit{Эмоция} &  angry & neutral & positive & sad & $\sum$ \\
		\hline
		$\mathrm{Precision_c}$ & 0.367 & 0.378 & 0.303 & 0.365 & 0.353 \\
		\hline
	\end{tabular}
\end{table}
Полнота системы -- это доля найденных классфикатором элементов принадлежащих классу относительно всех элементов этого класса в тестовой выборке. Рассчитывается согласно \ref{eq:recall}:
\begin{equation}\label{eq:recall}
	\mathrm{Recall_c} = \cfrac{A_{c,\,c}}{\displaystyle\sum_{i = 1}^{n}A_{i,\,c}},
\end{equation}
де $A$ -- матрица несоответствий, $c$ -- индекс класса для которого вычисляется полнота в матрице несоответствий. Значение полноты для каждого класса разметки представлено в таблице \ref{tab:recall}.

\begin{table}[H]
	\centering
	\caption{Значение точности для каждого класса разметки}\label{tab:recall}
	\begin{tabular}{|E|Q|Q|Q|Q|Q|}
		\hline
		\textit{Эмоция} &  angry & neutral & positive & sad & $\sum$ \\
		\hline
		$\mathrm{Recall_c}$ & 0.443 & 0.117 & 0.503 & 0.301 & 0.341 \\
		\hline
	\end{tabular}
\end{table}
F-мера представляет собой гармоническое среднее между точностью и полнотой и вычисляется согласно  \ref{eq:f1}:
\begin{equation}\label{eq:f1}
	\mathrm{F} = 2 \cdot \cfrac{\mathrm{Precision} \cdot \mathrm{Recall}}{\mathrm{Precision} + \mathrm{Recall}}
\end{equation}
Значение F-меры для каждого класса представлено в таблице \ref{tab:F}.
\begin{table}[H]
	\centering
	\caption{Значение F-меры для каждого класса разметки}\label{tab:F}
	\begin{tabular}{|E|Q|Q|Q|Q|Q|}
		\hline
		\textit{Эмоция} &  angry & neutral & positive & sad & $\sum$ \\
		\hline
		$\mathrm{F}_{c}$ & 0.366 & 0.281 & 0.432 & 0.321 & 0.347 \\
		\hline
	\end{tabular}
\end{table}
Наиболее высокое значение F-меры наблюдается у эмоций <<ярость/раздражение>> и <<позитив>>, наиболее низкое -- у классов <<нейтраль>> и <<грусть>>, причем разница с классом <<нейтраль>> наблюдается почти в 1.5 раза, с классом <<грусть>> -- в 1.4 раза. Такое отклонение может быть аргументировано следующим образом:
\begin{itemize}
	\item в обучающей выборке классов с низкой точностью распознавания используется слишком много коротких аудиозаписей (меньше секунды), и значимая часть интонационного рисунка (модуляция голоса и акцентуация при произнесении) фразы становится маловероятной;
	\item в обучающей выборке классов с низкой точностью распознавания используется слишком много длинных аудиозаписей, вмещающих несколько фраз, обладающих собственным интонационным контуром;
	\item неточная или неоднозначная разметка классов;
	\item эмоции <<ярость/раздражение>> и <<позитив>> имеют  более разнообразные или выраженные характеристики, что способствует лучшей их идентификации.
\end{itemize}
В таблице \ref{tab:dur} представлено количество аудиозаписей в трех диапазонах длительности:  Менее 1.5 секунд, 1.5-2.5 секунд и более 2.5 секунд. 
\begin{table}[H]
	\centering
	\caption{Длительности аудиозаписей по классам разметки}\label{tab:dur}
	\begin{tabular}{|E|F|F|F|}
		\hline
		\textit{Эмоция} & \textit{Менее 1.5 секунд, шт.} & \textit{1.5-2.5 секунд, шт.} & \textit{Более 2.5 секунд, , шт.} \\ 
		\hline
		angry   & 20  & 603 & 877 \\ \hline
		neutral & 20  & 622 & 858 \\ \hline
		positive & 14 & 597 & 889 \\ \hline
		sad     & 7  & 479 & 1014 \\ 
		\hline
	\end{tabular}
\end{table}
Можно заметить, что в классе <<грусть>> преобладают длинные аудиозаписи, содержащие, скорее всего, несколько фраз. Однако, длины в классе <<нейтраль>> распределены примерно так же как и в классах с высокими показателями распознавания.

Можно также предположить, что для качественного распознавания безэмоциональной речи требуется больший объем данных. В таблице \ref{tab:f-train} представлены замеры качества обучения классификатора с различными размерами обучающей выборки.

\begin{table}[htbp]
	\centering
	\caption{Замеры качества обучения классификатора}\label{tab:f-train}
	\renewcommand{\arraystretch}{1.3}
	\begin{tabular}{|F|E|E|E|E|}
		\hline
		\textit{Количество} & \multicolumn{4}{G|}{\textit{F-мера}} \\ \cline{2-5} 
		\textit{элементов, шт.} & angry & neutral & positive & sad \\ \hline
		80 & 0 & 0.196 & 0.34 & 0.254 \\ \hline
		240 & 0.379 & 0.179 & 0.260 & 0.215 \\ \hline
		400 & 0.413 & 0.270 & 0.334 & 0.218 \\ \hline
		560 & 0.321 & 0.140 & 0.387 & 0.265 \\ \hline
		800 & 0.340 & 0.145 & 0.344 & 0.308 \\ \hline
		1200 & 0.366 & 0.281 & 0.432 & 0.321 \\ 
		\hline
	\end{tabular}
\end{table}
Можно заметить, что F-мера каждого класса растет с увеличением количества элементов обучающей выборки.  

\section{Зависимость времени классификации от объема обучающей выборки}
\subsection{Замеры времени обучения классификатора}

ЭВМ, на которой проводились исследования производительности, обладает следующими техническими характеристиками:
\begin{itemize}
	\item операционная система EndeavourOS Linux x86\_64;
	\item оперативная память 8 Гб;
	\item процессор 11th Gen Intel i5-1135G7 (8) @ 4.200GHz.
\end{itemize}

В таблице \ref{tab:time-train} представлены результаты выполнения замеров скорости обучения классификатора на выборках различной длины. 
\begin{table}[H]
	\centering
	\caption{Замеры времени обучения классификатора}\label{tab:time-train}
	\renewcommand{\arraystretch}{1.2}
	\begin{tabular}{|H|H|}
		\hline
		\textit{Количество элементов обучающей выборки, шт} & \textit{Время обучения, мс} \\ \hline
		80 & 32333\\ \hline
		240 & 99034 \\ \hline
		400 & 121426 \\ \hline
		560 & 121696 \\ \hline
		800 & 121728 \\ \hline
		1200 & 122216 \\ \hline
	\end{tabular}
\end{table}
На рисунке \ref{fig:time-train} представлены данные из таблицы \ref{tab:time-train} в виде графика.
\begin{figure}[H]
	\centering
	\begin{tikzpicture}
	\begin{axis}[
		xlabel=Время обучения (мс),
		ylabel=Количество элементов (шт),
		axis lines=left,
		xmin=30000, xmax=130000,
		ymin=0, ymax=1300,
		grid = both,
		grid style = {dashed, lightgray!35},
		xtick distance = 10000,
		ytick distance = 200,
		width = 0.98\textwidth,
		tick label style={font=\scriptsize},
		scaled ticks=false,
		height=0.3\textheight,]
		]
		
		\addplot [
			color=blue,
			mark=square,
		] coordinates {
			(32333, 80)
			(99034, 240)
			(121426, 400)
			(121696, 560)
			(121728, 800)
			(122216, 1200)
		};
	\end{axis}
	\end{tikzpicture}
	\caption{График зависимости времени обучения от размера тренировочной выборки}
	\label{fig:time-train}
\end{figure}

\section*{Вывод}\addcontentsline{toc}{section}{Вывод}
Классификатор на выборке из 6000 элементов (1200 элементов в тестовой выборке и 300 в тренировочной для каждой эмоции) показал следующие результаты: точность -- 0.353, полнота -- 0.341 и F-мера -- 0.347.    
При расчете этих параметров для каждого класса было выявлено, что наиболее высокое значение F-меры наблюдается у эмоций <<ярость/раздражение>> и <<позитив>>, наиболее низкое -- у классов <<нейтраль>> и <<грусть>>, причем разница с классом <<нейтраль>> наблюдается почти в 1.5 раза, с классом <<грусть>> -- в 1.4 раза. Такое отклонение, скорее всего, вызвано тем, что для качественной характеристики наиболее <<ярких>> эмоций, то есть тех, которые имеют наиболее выраженные характеристики, требуется меньший объем обучающих данных. Также было выявлено, что на качество распознавания может влиять количество самостоятельных фраз в аудиозаписи, поскольку каждая фраза имеет свой интонационный контур, и повторение интонационных контуров в аудиозаписи может привести к ухудшению результатов.