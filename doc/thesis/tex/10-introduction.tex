\part*{ВВЕДЕНИЕ}
\addcontentsline{toc}{part}{\textbf{ВВЕДЕНИЕ}}
Человеческая речь является одними из самых естественных средств общения. Она содержит широкий спектр разнообразной информации, в том числе и эмоциональной. Для информационной системы важно не только понимать эмоции, но и их различать: например, исследование \cite{hsee2008feeling} показало, что  различные эмоции могут влиять на покупательное поведение по-разному. Например, чувство счастья может способствовать большей трате денег, в то время как чувство грусти может привести к сокращению трат. Более того, исследование показало, что люди, испытывающие страх или тревогу, часто проявляют склонность к покупке товаров, которые обеспечивают им защиту или безопасность.

Если в межличностной коммуникации большинство людей справляется с распознаванием речевых эмоций, то искусственные системы этому нужно обучить. Созданием технологий, ответственных за обработку эмоциональной информации в информационных системах, занимается направление, получившее название <<аффективные>> или <<эмоциональные вычисления>> (\textit{англ. affective computing}). В области аффективных вычислений существует множество техник обработки речевого сигнала и моделей классификации эмоций. 


Цель настоящей работы -- разработать метод определения эмоций по звучащей речи на основе скрытой марковской модели. Для достижения поставленной цели необходимо выполнить следующие задачи:
\begin{itemize}
	\item провести анализ существующих эмоциональных корпусов и выбрать наиболее подходящий для обучения классификатора;
	\item провести обзор информативных признаков, характеризующих речь и способов их выделения;
	\item провести обзор классификаторов, используемых в анализе речевых эмоций;
	\item спроектировать и реализовать метод детектирования эмоций;
	\item определить качественные характеристики классификатора, реализованного в рамках метода.
\end{itemize}