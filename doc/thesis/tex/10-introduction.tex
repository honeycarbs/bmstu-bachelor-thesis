\part*{ВВЕДЕНИЕ}
\addcontentsline{toc}{part}{\textbf{ВВЕДЕНИЕ}}
Человеческая речь является одними из самых естественных средств общения. Она содержит широкий спектр разнообразной информации, в том числе и эмоциональной. Знания об эмоции, которую испытывает человек в данный момент, могут быть использованы во многих сферах HCI, начиная от улучшения качества обслуживания и повышения покупательской способности и заканчивая повышением эффективности коммуникации и оказанием психологической помощи. 

Для информационной системы важно не только понимать эмоции, но и их различать: например, исследование \cite{hsee2008feeling} показало, что  различные эмоции могут влиять на покупательное поведение по-разному. Например, чувство счастья может способствовать большей трате денег, в то время как чувство грусти может привести к сокращению трат. Более того, исследование показало, что люди, испытывающие страх или тревогу, часто проявляют склонность к покупке товаров, которые обеспечивают им защиту или безопасность.

Если в межличностной коммуникации большинство людей справляется с распознаванием речевых эмоций, то искусственные системы этому нужно обучить. Созданием технологий, ответственных за обработку эмоциональной информации в информационных системах, занимается направление, получившее название <<аффективные>> или <<эмоциональные вычисления>> (\textit{англ. affective computing}). В области аффективных вычислений существует множество техник обработки речевого сигнала и моделей классификации эмоций. В настоящей работе будет рассмотрен метод определения речевых эмоций, использующих скрытые марковские модели -- статистический классификатор, который используется для моделирования последовательностей. 