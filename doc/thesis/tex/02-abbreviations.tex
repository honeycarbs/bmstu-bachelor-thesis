\part*{ОПРЕДЕЛЕНИЯ, ОБОЗНАЧЕНИЯ И\\СОКРАЩЕНИЯ}

В настоящей расчетно-пояснительной записке применяют следующие термины с соответствующими определениями.

\begin{enumdescript}
	\item[HCI] (\textit{англ. human-computer interaction}) -- область науки, которая изучает взаимодействие между человеком и компьютером, а также создает и улучшает интерфейсы, которые облегчают этот процесс. 
	\item[Эмоция] -- психическое состояние, которое включает в себя такие элементы, как субъективные ощущения, физиологические изменения и выражается через мимику, жесты и голос. 
	\item[Речь] -- процесс создания звуковых волн, которые возникают в результате вибрации голосовых связок в гортани, и затем формируются в узнаваемые звуки и слова при помощи движений губ, языка, неба и других артикуляционных органов.
	\item[Фреймы] -- отрезки аудиосигнала длительности как правило 10-40 мс, идущие <<внахлест>>, то есть таким
	образом, чтобы начало очередного фрейма пересекалось с концом предыдущего. 
%	\item[Паралингвистически признаки речи] -- невербальные признаки, передающие совместно с вербальными смысловую информацию в составе речевого сообщения.
	\item[Частота основного тона] -- частота колебания голосовых связок при произнесении тоновых звуков.
	\item[Джиттер] -- мера возмущений частоты основного тона, показывающая непроизвольные изменения в частоте смежных вибрационных циклов голосовых складок.
	\item[Шиммер] -- мера аналогичная джиттеру, только характеризующая пертурбации амплитуд сигнала на смежных циклах колебаний основного тона.
	\item[Форманты] --  пики в огибающей спектра звука, создаваемые акустическими резонансами в голосовом тракте. 
	\item[Кепстр] -- преобразование Фурье от логарифма спектра мощности.
	\item[Мел] -- единица измерения частоты звука, основанная на статистической обработке большого числа данных о субъективном восприятии высоты звуковых тонов.
%	\item[Редукция] (\textit{в лингвистике}) -- ощущаемое человеческим ухом изменение звуковых характеристик речевых элементов, вызванное их безударным положением по отношению к ударным элементам.
	\item[Кластеризация] -- разделение множества входных векторов данных на кластеры (группы) по степени схожести друг с другом.
	\item [Матрица несоответствий] (\textit{англ. confusion matrix}) -- таблица, в которой каждая строка соответствует истинному классу объектов, а каждый столбец -- прогнозируемому классу. В ячейках таблицы указано количество аудиофайлов, которым в результате алгоритма распознавания был присвоен тот или иной класс.
\end{enumdescript}
