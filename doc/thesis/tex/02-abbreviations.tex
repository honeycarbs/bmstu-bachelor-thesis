\part*{ОПРЕДЕЛЕНИЯ, ОБОЗНАЧЕНИЯ И\\СОКРАЩЕНИЯ}

В настоящей расчетно-пояснительной записке применяют следующие термины с соответствующими определениями.

\begin{enumdescript}
%	\item[Эмоция] --
%	\item[Речь] -- 
	\item[Фреймы] -- отрезки аудиосигнала длительности как правило 10-40 мс, идущие <<внахлест>>, то есть таким
	образом, чтобы начало очередного фрейма пересекалось с концом предыдущего. 
	\item[Мел] -- единица измерения частоты звука, основанная на статистической обработке большого числа данных о субъективном восприятии высоты звуковых тонов.
	\item[Периодограмма] -- функция от частоты, которая показывает оценку спектральной плотности сигнала.
	\item[Форманты] --  пики в огибающей спектра звука, создаваемые акустическими резонансами в голосовом тракте. 
\end{enumdescript}
