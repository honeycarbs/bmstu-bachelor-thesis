\usepackage[T1,T2A]{fontenc}
\usepackage[utf8]{inputenc}
\usepackage[english,main=russian]{babel}
%\usepackage{fix-cm}

\usepackage{lmodern}
\DeclareFontFamilySubstitution{T2A}{lmr}{cmr}
\DeclareFontFamilySubstitution{T2A}{lmss}{cmss}
\DeclareFontFamilySubstitution{T2A}{lmtt}{cmtt}

\usepackage{microtype}
\microtypesetup{expansion=false}
\sloppy

\usepackage[
	left=30mm,
	right=10mm, 
	top=20mm,
	bottom=20mm,
]{geometry}

\makeatletter
	\renewcommand\LARGE{\@setfontsize\LARGE{22pt}{20}}
	\renewcommand\Large{\@setfontsize\Large{20pt}{20}}
	\renewcommand\large{\@setfontsize\large{16pt}{20}}
\makeatother

\usepackage{microtype} % Настройка переносов
\sloppy

\usepackage{setspace} % Настройка межстрочного интервала
\onehalfspacing

\usepackage{indentfirst} % Настройка абзацного отступа
\setlength{\parindent}{12.5mm}

\usepackage[unicode,hidelinks]{hyperref}
\usepackage{xifthen}

\usepackage{colortbl}

\usepackage[normalem]{ulem}
% Текст под линией 
\newcommand*{\undertext}[2]{%
	\begin{tabular}[t]{@{}c@{}}%
		#1\\\relax(\scriptsize#2)%
	\end{tabular}
}

% горизонтальная линия
\makeatletter
\newcommand{\vhrulefill}[1]
{
	\leavevmode\leaders\hrule\@height#1\hfill \kern\z@
}

\usepackage[figure,table]{totalcount} % Подсчет изображений, таблиц
\usepackage{rotating} % Поворот изображения вместе с названием
\usepackage{lastpage} % Для подсчета числа страниц

\usepackage{titlesec}
\usepackage{titletoc}

\newcommand{\makechapterdots}{
	\titlecontents{chapter}[0pt]
	{}
	{\large\thecontentslabel. \enspace}
	{}
	{\titlerule*[1pc]{.}\contentspage}
}

\newcommand{\removechapterdots}{
	\titlecontents{chapter}[0pt]
	{}
	{\large\thecontentslabel. \enspace}
	{}
	{\titlerule*[1pc]{}\contentspage}
}


\titlecontents{section}[20pt]
{}
{\large\thecontentslabel. \enspace}
{}
{\titlerule*[1pc]{.}\contentspage}

\titlecontents{subection}[20pt]
{}
{\large\thecontentslabel. \enspace}
{}
{\titlerule*[1pc]{.}\contentspage}

\titlecontents{subsubection}[20pt]
{}
{\large\thecontentslabel? \enspace}
{}
{\titlerule*[1pc]{.}\contentspage}


\titleformat{\chapter}[block]
{\large\scshape\bfseries}{\thechapter.}{0.5em}{\large\scshape}

\titleformat{\section}[block]
{\hspace{\parindent}\large\scshape\bfseries}{\thesection.}{0.5em}{\large\scshape\raggedright}

\titleformat{\subsection}[block]
{\hspace{\parindent}\large\scshape\bfseries}{\thesubsection.}{0.5em}{\large\scshape\raggedright}

\titleformat{\subsubsection}[block]
{\hspace{\parindent}\Large\scshape\bfseries}{\thesubsection.}{0.5em}{\large\scshape\raggedright}

\titlespacing{\chapter}{12.5mm}{-22pt}{10pt}
\titlespacing{\section}{12.5mm}{10pt}{10pt}
\titlespacing{\subsection}{12.5mm}{10pt}{10pt}
\titlespacing{\subsubsection}{12.5mm}{10pt}{10pt}

\usepackage{tocloft}
\renewcommand{\cftsubsecfont}{\normalfont}  
\renewcommand{\cftsubsecaftersnum}{.}
\renewcommand{\cftsecaftersnum}{.}

% ---------------------------------------- CAPTION --------------------------------- %

\usepackage[
	labelsep=endash,
	singlelinecheck=false,
]{caption}

\captionsetup[figure]{justification=centering}
\captionsetup[table]{justification=raggedleft}
\captionsetup[listing]{justification=raggedright}


% ---------------------------------------- ABBRS --------------------------------- %

\usepackage{enumitem}
\newcounter{descriptcount}
\newlist{enumdescript}{description}{2}
\setlist[enumdescript,1]{%
	before={\setcounter{descriptcount}{0}%
		\renewcommand*\thedescriptcount{\arabic{descriptcount})}}
	,font=\scshape\stepcounter{descriptcount}\thedescriptcount~
}
\setlist[enumdescript,2]{%
	before={\setcounter{descriptcount}{0}%
		\renewcommand*\thedescriptcount{\roman{descriptcount}}}
	,font=\scshape\stepcounter{descriptcount}\thedescriptcount~
}

\def\labelitemi{--} % Изменение буллета для списков


% ---------------------------------------- TABLE  ----------------------------------------

\usepackage{xcolor}
\usepackage{tabularx}
\usepackage{booktabs}
\usepackage{multirow}

% ---------------------------------------- FIGURE ----------------------------------------

\usepackage{graphicx}
\usepackage{float}
\usepackage{wrapfig}
\usepackage{tikzscale}
\usepackage[notransparent]{svg}

\usepackage{pgfplots}
\pgfplotsset{compat=newest}

% ----------------------------------------- MATH -----------------------------------------

\RequirePackage{lscape}
\RequirePackage{afterpage}

\RequirePackage{amsmath}
\RequirePackage{amssymb}

% ----------------------------------------- BIBLIO ---------------------------------------

\usepackage[
	style=gost-numeric,
	language=auto,
	autolang=other,
	sorting=none,
	maxbibnames=99,
	backend=biber,
]{biblatex}
\usepackage{csquotes}

\DefineBibliographyStrings{russian}{%
	urlfrom = {Режим доступа},
}

\DeclareFieldFormat{urldate}{(дата обращения:\addspace\thefield{urlday}\adddot \thefield{urlmonth}\adddot\thefield{urlyear})}

\DeclareFieldFormat{url}{\bibstring{urlfrom}\addcolon\space\url{#1}}
