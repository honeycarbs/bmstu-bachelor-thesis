\chapter*{ОПРЕДЕЛЕНИЯ, ОБОЗНАЧЕНИЯ И\\СОКРАЩЕНИЯ}

В настоящей расчетно-пояснительной записке применяют следующие термины с соответствующими определениями.

\begin{enumdescript}
	\item[Речевой звук] -- кратчайшая, далее неделимая единица языка.
	\item[Фонема] -- необходимый и достаточный звуковой минимум для конституирования морфемы, точнее, экспонента последней. 
	\item[Аллофон] -- Вариант фонемы, обусловленный конкретным фонетическим окружением, т.~е. оттеночная фонема.
	\item[Дифон] -- соседняя пара фонем в высказывании.
	\item[Трифон] --  последовательность из трех соседних фонем. 
	\item[Джиттер] --  мера возмущений частоты основного тона, показывающая непроизвольные изменения в
	частоте смежных вибрационных циклов голосовых складок.
	\item[Шиммер] -- мера, характеризующая возмущениями амплитуд сигнала на смежных циклах колебаний основного тона.
	\item[Вокализация] --  отношение количества вокализованных к количеству невокализованных кадров.
	\item[Мел] -- единица высоты звука, основанная на восприятии этого звука органами слуха.
	\item[Гребенчатый фильтр, гребенка фильтров] --  фильтр, реализованный путем добавления версии с задержкой сигнала самому себе, вызывая конструктивную и деструктивную помеху. АЧХ гребенчатого фильтра состоит из серии регулярно расположенных меток, что создает вид гребенки.
	\item[Окно] --  анализируемый сегмент сигнала.
\end{enumdescript}