\chapter*{ОПРЕДЕЛЕНИЯ, ОБОЗНАЧЕНИЯ И СОКРАЩЕНИЯ}

В настоящей расчетно-пояснительной записке применяют следующие термины с соответствующими определениями.

\begin{enumdescript}
	\item[Речевой звук] -- кратчайшая, далее неделимая единица языка.
	\item[Артикулярный тракт] -- совокупность огранов человеческого тела,  которые используются в процессе голосоведения, и в первую очередь в процессе речепроизводства (лёгкие, гортань,  глотка, ротовая и носовая полости, увула, мягкое и твердое небо, язык, зубы, губы). 
	\item[Голосовой источник звука] -- периодическая модуляция голосовыми складками воздушного потока, подаваемого из лёгких.
	\item[Шумовой систочник звука] -- генерация шума турбулентными завихрениями воздушного потока в сужениях речеобразующего аппарата.
	\item[Импульсный источник звука] -- скачкообразное изменение давления воздуха при резком открытии смычки в артикулярном тракте.
	\item[Турбулентный поток] -- поток тре­ния воздушной струи, вы­званный су­же­ни­ем ар­тику­ли­рую­щих ор­га­нов.
	\item[Маскировка] -- повышение порога слышимости звука (стимула) в присутствии других звуков (маскеров). Маскирующие звуки могут предшествовать (прямая), действовать одновременно (одновременная) и следовать за сигналом (обратная маскировка).
%	\item[Смычные согласные] -- звуки, при артикуляции которых органы речи находятся в таком положении, что поток воздуха из лёгких полностью блокируется с помощью смычки, создаваемой в полости рта или в гортани.
%	\item[Сонорные] --
%	\item[Щелевые] --
%	\item[Аффикаты] -- согласные, представляющие собой слитное сочетание смычного согласного с фрикативным, обычно того же места образования. Например,  <<ч>> (из <<т>> + <<щ>>)
%	\item[Взрывные] --

\end{enumdescript}