\chapter*{ВВЕДЕНИЕ}
\addcontentsline{toc}{chapter}{\textbf{ВВЕДЕНИЕ}}

Идентификация эмоционального состояния на основе характеристик речевого сигнала вызывает интерес как со стороны научного сообщества, так и со стороны многих коммерческих организаций. Каждая такая система имеет некоторые задачи, которые она призвана решать, и комплекс подходов, которые применяются для решения поставленных задач.

На сегодняшний день в когнитивистике и исследованиях связанных с искусственным интелектом происходит интенсивное развитие информационных технологий, которые вносят качественное улучшение при взаимодействии между субъектами в системах взаимодействия <<человек-компьютер>> и <<человек-человек>>. 

В системе <<человек-компьютер>> системы идентификации эмоционального состояния находят применение при коммуникации людей c голосовыми помощниками, эмоциональном окрашивании речи операторов автоматизированных колл-центров, улучшении систем виртуальной реальности и.~т.~д. В системе <<человек-человек>> появляется возможность уточнения автоматизированного перевода, улучшения систем детектирования лжи, более точной диагностики психических расстройств на основе изменения эмоционального фона за период времени, мониторинга настроения толпы.


Целью настоящей работы является классификация способов решения задачи разпознавания эмоций из звучащей речи.
Для достижения поставленной цели требуется решить ряд задач:
\begin{itemize}
	\item сформировать классификацию эмоциональных состояний;
	\item провести обзор информативных признаков, характеризующих речь;
	\item описать формальную постановку задачи;
	\item описать существующие решения задачи.
\end{itemize}
% Для этого необходимо провести обзор существующих методов извлечения эмоционального состояния на основе индивидуальных характеристик речи и классифицировать способы решения поставленной задачи.


