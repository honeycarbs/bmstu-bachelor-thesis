\chapter*{РЕФЕРАТ}
\addcontentsline{toc}{chapter}{\textbf{РЕФЕРАТ}}

Расчетно--пояснительная записка \pageref{LastPage} с., \totalfigures\ рис., \totaltables\ табл., 21 ист., 0 прил.

Представлена категоризация эмоциональных пространств. Классифицированы способы решения задачи распознавания эмоций по звучащей речи. Проведен обзор информативных признаков, характеризующих речь. Рассмотрены некоторые акустические характеристики, а именно: просодические, спектральные, динамические. Описана формальная постановка рассматриваемой задачи. Приведены примеры классификаторов, используемых при решении задачи и описан принцип их работы.